\documentclass[12pt, oneside, letterpaper]{report}
\usepackage{ifpdf}
\usepackage[colorlinks,bookmarksopen]{hyperref}
\usepackage{layout}
\usepackage{lscape}
\textwidth=475pt
\hoffset=-.5in
\begin{document}
% ----------------------------------------------- front matter of the document
%\maketitle
\part{System and Software Design Description (SSDD): Incorporating
     Architectural Views and Detailed Design Criteria \\ for \\ Groups in a
     Univeristy Setting}
\tableofcontents                                % chapter with the table of contents
\listoffigures
\listoftables
%------------------------------------------------ body of the document
\chapter{Introduction}
	\section{Identification}
		The software system being considered for development is referred
		to as Groups in a University Setting or gus. The customer providing
		specifications for the ethnic and religion team is the Lutheran
		Campus Ministry. The ultimate customer, or end-user, of the system
		will be student groups at the University of Idaho. This is a new
		project effort, so the version under development is version 0.05
	\section{Purpose}
		The purpose of the system under development is to provide a tool
		for the easy administration and tracking of university-style groups
		including but not limited to clubs and sports teams, the system will
		also try to increase student involvement by connecting and recognizing
		the involvement of users. While the system will be used by university
		personnel, this document is intended to be read and understood by UICS
		software designers and coders.  This document will also be approved by
		Dr. Clinton Jeffery.
	\section{Scope}
		GUS is to social-networking as an intranet is to the Internet.
		Where other social networks distract users with non-university
		non-local non-face-to-face non-involvement, GUS will focus these
		types of functionalities to the university and local community
		setting to best meet the administrative and service needs of groups
		in a university setting.  In addition to being a group-centered
		student-involvement web-application.
	\section{Definitions, Acronyms, and Abbreviations}
		\begin{tabular}{|p{4cm}|p{10cm}|}
		\hline
		\textbf{Term or Acronym} & \textbf{Definition} \\ \hline
		Alpha test & Limited release(s) to selected, outside testers \\ \hline
		Beta test & Limited release(s) to cooperating customers wanting early access to developing systems \\ \hline
		Final test & aka, Acceptance test, release of full functionality to customer for approval \\ \hline
		DFD & Data Flow Diagram \\ \hline
		SDD & Software Design Document, aka SDS, Software Design Specification \\ \hline
		SRS &  Software Requirements Specification \\ \hline
		SSRS & System and Software Requirements Specification \\ \hline
		GUS & Groups in a University Setting \\ \hline
		\end{tabular}
	\section{References}
		\begin{enumerate}
			\item www.churchteams.com
			\item www.groupmeister.com
			\item www.teamr.com
			\item www.salesboom.com
			\item www.wikipedia.org
		\end{enumerate}
	\section{Overview and Restrictions}

	This document is for limited release only to UI CS personnel
	working on the project.

		Section 2 of this document describes the system under development
		from a holistic point of view.  Functions, characteristics,
		constraints, assumptions, dependencies, and overall requirements
		are defined from the system-level perspective.

		Section 3 of this document describes the specific requirements of
		the system being developed.  Interfaces, features, and specific
		requirements are enumerated and described to a degree sufficient
		for a knowledgeable designer or coder to begin crafting an
		architectural solution to the proposed system.

		Section 4 provides the requirements traceability information for
		the project.  Each feature of the system is indexed by the SSRS
		requirement number and linked to its SDD and test references.

		Sections 5 and up are appendices including original information
		and communications used to create this document.
\nopagebreak
\chapter{Constraints and Stakeholder Concerns}
	\section{Constraints}
		\subsection{Environmental Constraints}
		\subsection{System Requirement Constraints}
		\subsection{User Characteristic Constraints}
	\section{Stakeholder Concerns}
\chapter{System and Software Architechture}
	\section{Users Architectural View}
		\subsection{User's View Identification}
		\subsection{User's View Representation and Description}
	\section{Developer's Architectural View}
		 \subsection{Developer's View Identification}
		 \subsection{Developer's View Representation and Description}
			\subsubsection{Object Model}
			\subsubsection{Dynamic Model}
		 \subsection{Developer's Architectural Rationale}
	\section{Consistency of Architecutural Views}
		\subsection{Developer's Viewpoint Detailed Software Design}
		\subsection{Component Dictionary}
		\subsection{Component Detailed Design}
			\subsubsection{Detailed Design for Component/Entity: Name of Component}
			\subsubsection{Detailed Design for Component/Entity: Name of Component}
			\subsubsection{Detailed Design for Component/Entity: Name of Component}
			\subsubsection{Detailed Design for Component/Entity: Name of Component}
	\section{Data Dictionary}
\chapter{Requirements Traceability}
\part{Systems and Software Requirements Specification (SSRS) \\ for \\ Groups in a University Setting}
\chapter{Overall Description}
	\section{Product Perspective}
		Gus is an independent software system, as it does not directly
		integrate with a larger system. However, GUS does draw data from
		external sources, such as personal information databases, and
		needs to be integrated with a web server in order to be readily
		accessible.
	\section{Product Functions}
		\begin{enumerate}
		\item Simplifying tasks to leaders of groups, such as:
			\begin{enumerate}
				\item Sending notifications to group members, prospective
					members, former members, and interested community
					members (email)
				\item Sending information (files) to group members via
					email or download link
				\item Managing a group-wide calendar of events
				\begin{enumerate}
					\item track volunteers, attendees, and contributors
					\item suggest potentially beneficial services other
						groups could provide related to the event
					\item provide a calendar of events that includes events
						from other groups that members would want to attend
						(like marching band if half of LCM are in the
						marching	band.)
					\item keeping track of who is responsible for bringing
						/ doing what at an event
				\end{enumerate}
				\item Automatically generating:
				\begin{enumerate}
					\item Contact information (contact sheets, phone
						directories)
					\item Website with updated contact, group, event, and
						customized information
					\item Organization charts
					\item Graphical relationships between groups
					\item Fees, dues, and expenses notifications
					\item Event reminders
				\end{enumerate}
			\end{enumerate}
			\item Consolidating information for members, former members,
				potential members (and parents) of groups:
			\begin{enumerate}
				\item Common location of group information
				\item Searching existing groups
				\item Tying together existing groups (even suggesting
					similar groups)
				\item Personalized emails regarding changes/updates
				\item Outstanding expenses or reimbursements
				\item Reliable (i.e., automatically updated):
				\begin{enumerate}
					\item Group contact information
					\item Group event information
				\end{enumerate}
				\item Transcript of verified group activity (for use with
					service-learning classes, and proof of volunteerism
					for potential employers)
				\item Supplementing Vandal Friday with emails to
					prospective high school Seniors
			\end{enumerate}
			\item Getting member input though: forums, project managers,
				surveys, and polls
			\item Payment processing and sponsorship collection
			\item 4. Recruitment and advertising for groups, volunteer
				/ paid opportunities, services provided, possibly a
				bartering tool
		\end{enumerate}
	\section{User Characteristics}
		Gus should be easy for any user to understand with a brief
		explanation and intuitive enough for an uninitiated user to
		figure out by looking through the options. Basic computer use
		skills and a simple conceptual explanation should be enough for
		every day usage.
	\section{Constraints}
		GUS must meet privacy policies as they apply to both the
		University of Idaho and social networking sites.  GUS must be
		able to interface with outside database servers (such as the
		Center for Volunteerism's database, UI's career seeker site,
		common social networks, and parent groups of university groups).
		Member's activities and group's activities must be audited for
		accuracy and safety.  The languages used to program GUS will be
		primarily, HTML, CSS, and PHP for the user interface, C++ for
		the interface between the user interface (which will implement
		security and complex business rules), and the database, and
		SQL for the database.  The networking protocols will be TCP/IP
		and Open MP / MPI will be used to enhance parallel operation.
		The system will have personal information for over 5,000
		students, so confidentiality is of the utmost importance.
	\section{Assumptions and Dependencies}
		The software system should run like a web-app and need not be
		downloaded by users.  It is assumed that users will be running
		Internet Explorer, Fire Fox, or another popular web browser.
		The server for the system is expected to run a UNIX operating
		system.
	\section{System Level (Non-Functional) Requirements}
		\subsection{Site dependencies}
			GUS will require a server that can support 1,000 concurrent
			users.  The database must store the information, interests,
			and activities of approximately 5,000 external users, 5,000
			students and 200 groups.
		\subsection{Safety, security and privacy requirements}
			GUS contains the personal information of over 5,000 users
			security should be integrated into every facet of this
			program.  The privacy criteria for this system must reflect
			privacy policies that apply to the University of Idaho, and
			the security criteria for this system must reflect the need
			to secure over 5,000 users from identity theft and potential
			defamation of character.
		\subsection{Performance requirements}
		\begin{enumerate}
			\item The number of simultaneous users to be supported are: 1,000.
			\item Supported information ranges from text to files to streaming video.
			\item 95\% of the transactions shall be processed in less than half a second.
		\end{enumerate}
		\subsection{System and software quality}
			Gus must perform all required functions, behave consistently
			and correctly, be easily corrected, running between 5:30 am
			all day to 1:30 am be easily adaptable, test-driven, and easy
			to use.
		\subsection{Packaging and delivery requirements}
			The executable system and all associated documentation (i.e.,
			SSRS, SDD, code listing, test plan (data and results), and
			user manual) will be delivered to the customer via Internet
			download. The final, edited version of the above documents
			will accompany the final, accepted version of the executable system.
		\subsection{Personnel-related requirements}
			The system under development will require a graduate student
			system-level administrator to maintain the system.
		\subsection{Training-related requirements}
			No training materials or expectations are tied to this
			project other than the limited help screens built into the
			software and the accompanying user manual.
		\subsection{Logistics-related requirements}
			A server will be required to maintain the software system.
			The user will be required to have an Internet connection.
		\subsection{Precedence and criticality of requirements}
			\begin{enumerate}
				\item Maintaining confidentiality and privacy of PII
				\item This system must be reliable enough for users to not
					give up on it
				\item All other features are less important than the first
					two and equally important
			\end{enumerate}
\chapter{Specific Requirements}
	\section{External Interface Requirements}
		\subsection{Hardware Interfaces}
			The system will require a server and secure networking
			abilities.
		\subsection{Software Interfaces}
			The system will require an interface to interact with
			emailing systems, databases, and authentication servers.
		\subsection{User Interfaces}
			The system will require user interfaces for non-university
			users (prospective students, community members, alumni,
			parent groups, etc.), students, officers, and staff/faculty.
		\subsection{Other Communication Interfaces}
			 GUS will interface with the university career seeking site
			 and social networking sites.
			 \pagebreak
			 \begin{landscape}
		\begin{table} \caption{Hardware Interfaces}
		\begin{tabular}{|p{1.75cm}|p{4cm}|p{5.25cm}|p{2.5cm}|p{3cm}|p{1.75cm}|} \hline
								%% Titles
			 \textbf{Name}
			 & \textbf{Source/Destination}
			 & \textbf{Description}
			 & \textbf{Type/range}
			 & \textbf{Dependencies}
			 & \textbf{Formats}
			 \\\hline
			 %% row 2
			 HTTP Server
			 & Dedicated Server or VPS / Client
			 & This device is responsible for serving HTML content (and
			   other content) to clients. Preferably Apache2.
			 & All
			 & Requires a server-capable machine
			 & N/A
			 \\\hline
			 %% row 3
			 VPS or Dedicated Server
			 & NA
			 & A VPS or a Dedicated Server, preferably running a
			   preconfigured Linux distribution such as Fedora or Ubuntu.
			 & All
			 & Electricity, high-speed Internet connection
			 & N/A
			 \\\hline

		\end{tabular} \end{table}
		\begin{table} \caption{Software Interfaces}
		\begin{tabular}{|p{1.75cm}|p{4cm}|p{5.25cm}|p{2.5cm}|p{3cm}|p{1.75cm}|}
		\hline
								%% Titles
			 \textbf{Name}
			 & \textbf{Source/Destination}
			 & \textbf{Description}
			 & \textbf{Type/range}
			 & \textbf{Dependencies}
			 & \textbf{Formats}
			 \\\hline
			 %% row 2
			 SQL Server
			 & Dedicated Server or VPS / Client
			 & Works in conjunction with HTTP server to provide data.
			 & All
			 & Requires a server-capable machine
			 & N/A
			 \\\hline
			 %% row 3
			 PHP5
			 & HTTP Server / Client
			 & Provides computational power so tasks that serve HTML
			   content via apache can be completed.
			 & Requires a server capable of running PHP5.
			 & Electricity, high-speed Internet connection
			 & N/A
			 \\\hline
			 \end{tabular}
			 \end{table}
		\begin{table} \caption{User Interfaces}
		\begin{tabular}{|p{1.75cm}|p{4cm}|p{5.25cm}|p{2.5cm}|p{3cm}|p{1.75cm}|}
		\hline
								%% Titles
			 \textbf{Name}
			 & \textbf{Source/Destination}
			 & \textbf{Description}
			 & \textbf{Type/range}
			 & \textbf{Dependencies}
			 & \textbf{Formats}
			 \\\hline
			 %% row 2
			 Website
			 & HTTP Server/Client
			 & Allows user to interact with the service
			 & All
			 & HTTP Server
			 & Web
			 \\\hline
			 %% row 3
			 Cell phone
			 & Cellphone
			 & Receive text-messages
			 & All
			 & HTTP Server
			 & Text
			 \\\hline
			 \end{tabular}
			 \end{table}
			\begin{table} \caption{Other Communication Interfaces}
			\begin{tabular}{|p{1.75cm}|p{4cm}|p{5.25cm}|p{2.5cm}|p{3cm}|p{1.75cm}|}
			\hline
								%% Titles
			 \textbf{Name}
			 & \textbf{Source/Destination}
			 & \textbf{Description}
			 & \textbf{Type/range}
			 & \textbf{Dependencies}
			 & \textbf{Formats}
			 \\\hline
			 %% row 2

			 &
			 &
			 &
			 &
			 &
			 \\\hline
			 \end{tabular}
			 \end{table}
			 \end{landscape}
	\pagebreak
	\section{System Features}
		\subsection{Use Case Diagrams}
		insert 1+ use case diagrams here
		\subsection{System feature 1: Officer Management (Administration
			Module)}
			\subsubsection{Add Officer}
			\begin{itemize}
				\item{\textbf{Actors:} Group Administrator}
				\item{\textbf{Goals:} Administrator will add an officer
					to the group}
				\item{\textbf{Preconditions:} Administrator is logged onto
					to the group's administration page}
				\item{\textbf{Summary:} Related use cases: modify officer
					access}
				\item{\textbf{Steps:}
				\begin{enumerate}
					\item{Administrator clicks add officer button}
					\item{Administrator types in officer's username}
					\item{Administrator selects the officer's title/role}
					\item{Administrator modifies officer's access}
					\item{Administrator selects save and exit}
				\end{enumerate}
				}
				\item{\textbf{Alternatives:} Administrator cancels out of
					use case}
				\item{\textbf{Postconditions:} the selected user is given
					officer status in the group}
			\end{itemize}
			\subsubsection{Modify Officer Access}
			\begin{itemize}
				\item{\textbf{Actors:} Group Administrator}
				\item{\textbf{Goals:} Administrator will change the access given to an officer}
				\item{\textbf{Preconditions:} Administrator is logged onto to the group's administration page}
				\item{\textbf{Summary:} Related use cases:}
				\item{\textbf{Steps:}
				\begin{enumerate}
					\item{Administrator clicks modify officer access
						button}
					\item{Administrator selects officer / role to be
						modified}
					\item{Administrator selects type of access to be
						modified}
					\item{Administrator selects save and exit}
				\end{enumerate}
				}
				\item{\textbf{Alternatives:} Administrator cancels out of
					use case}
				\item{\textbf{Postconditions:} The selected officer(s)
					access is modified}
			\end{itemize}

			\subsubsection{Remove Officer}
			\begin{itemize}
				\item{\textbf{Actors:} Group Administrator}
				\item{\textbf{Goals:} Administrator will remove an officer
					from the group}
				\item{\textbf{Preconditions:} Administrator is logged onto
					to the group's administration page}
				\item{\textbf{Summary:} Related use cases:}
				\item{\textbf{Steps:}
				\begin{enumerate}
					\item{Administrator clicks remove officer button}
					\item{Administrator selects officer}
					\item{Administrator selects save and exit}
				\end{enumerate}
				}
				\item{\textbf{Alternatives:} Administrator cancels out of
					use case}
				\item{\textbf{Postconditions:} the selected user is
					removed as a group officer}
			\end{itemize}
			\subsubsection{Change Group Password}
			\begin{itemize}
				\item{\textbf{Actors:} Group Administrator}
				\item{\textbf{Goals:} Administrator will change the password to access the group page}
				\item{\textbf{Preconditions:} Administrator is logged onto to the group's administration page}
				\item{\textbf{Summary:} Related use cases:}
				\item{\textbf{Steps:}
				\begin{enumerate}
					\item{Administrator clicks change group password button}
					\item{Administrator types in the new password twice}
					\item{Administrator selects save and exit}
				\end{enumerate}
				}
				\item{\textbf{Alternatives:} Administrator cancels out of use case}
				\item{\textbf{Postconditions:} the group password is changed}
			\end{itemize}
			\subsubsection{Edit Forum Priveleges}
			\begin{itemize}
				\item{\textbf{Actors:} Group Administrator}
				\item{\textbf{Goals:} Administrator will select who can make changes to forums}
				\item{\textbf{Preconditions:} Administrator is logged onto to the group's administration page}
				\item{\textbf{Summary:} Related use cases:}
				\item{\textbf{Steps:}
				\begin{enumerate}
					\item{Administrator clicks forum priveleges button}
					\item{Administrator selects who's priveleges will be changed}
					\item{Administrator selects the the priveleges}
					\item{Administrator saves and exits}
				\end{enumerate}
				}
				\item{\textbf{Alternatives:} Administrator cancels out of use case}
				\item{\textbf{Postconditions:} the user's priveledges are changed}
			\end{itemize}
		\subsection{System feature 2: Group Management (Officer Module)}
			\subsubsection{Add Member}
			\begin{itemize}
				\item{\textbf{Actors:} Officer}
				\item{\textbf{Goals:} Officer will add a member to the group}
				\item{\textbf{Preconditions:} Officer is logged onto to the group's officer page}
				\item{\textbf{Summary:} Related use cases:}
				\item{\textbf{Steps:}
				\begin{enumerate}
					\item{Officer clicks add member button}
					\item{Officer types in the member's username}
					\item{Officer selects save and exit}
				\end{enumerate}
				}
				\item{\textbf{Alternatives:} Officer cancels out of use case}
				\item{\textbf{Postconditions:} the selected user is given membership status in the group}
			\end{itemize}
			\subsubsection{Remove Member}
			\begin{itemize}
				\item{\textbf{Actors:} Officer}
				\item{\textbf{Goals:} Officer will remove member from the group}
				\item{\textbf{Preconditions:} Officer is logged onto to the group's officer page}
				\item{\textbf{Summary:} Related use cases:}
				\item{\textbf{Steps:}
				\begin{enumerate}
					\item{Officer clicks remove member button}
					\item{Officer selects member}
					\item{Officer selects save and exit}
				\end{enumerate}
				}
				\item{\textbf{Alternatives:} Officer cancels out of use case}
				\item{\textbf{Postconditions:} the selected user is removed from the group}
			\end{itemize}
		\subsubsection{Add Event}
			\begin{itemize}
				\item{\textbf{Actors:} Officer}
				\item{\textbf{Goals:} Add an event to fit officer specifications}
				\item{\textbf{Preconditions:} Officer is logged onto to the group's officer page}
				\item{\textbf{Summary:} Related use cases: Modify Event}
				\item{\textbf{Steps:}
				\begin{enumerate}
					\item{Officer clicks Events button}
					\item{Officer clicks Add New Event}
					\item{Officer adds a title}
					\item{Officer chooses a date, time, and recurrence}
					\item{Officer checks the post to site option}
					\item{Officer checks the email notifications to members option}
					\item{Officer modifies the message to be sent}
					\item{Officer selects save and exit}
				\end{enumerate}
				}
				\item{\textbf{Alternatives:} Officer cancels out of use case by clicking on a different navigational area or clicking Cancel}
				\item{\textbf{Postconditions:} Event has been added to fit the officer's specifications}
			\end{itemize}
		\subsubsection{Modify Event}
			\begin{itemize}
				\item{\textbf{Actors:} Officer}
				\item{\textbf{Goals:} modify (add, edit, remove) an event on the calendar}
				\item{\textbf{Preconditions:} Officer is logged onto to the group's officer page}
				\item{\textbf{Summary:} Related use cases: Add Event, Remove Event}
				\item{\textbf{Steps:}
				\begin{enumerate}
					\item{Officer clicks Events button}
					\item{Officer clicks Modify Event button}
					\item{Officer modifies event information}
					\item{Officer sends notfications and has the site updated as needed}
					\item{Officer selects save and exit}
				\end{enumerate}
				}
				\item{\textbf{Alternatives:} Officer cancels out of use case by clicking on a different navigational area or clicking Cancel}
				\item{\textbf{Postconditions:} Event has been modified to fit the officer's specifications}
			\end{itemize}
			\subsubsection{Remove Event}
			\begin{itemize}
				\item{\textbf{Actors:} Officer}
				\item{\textbf{Goals:} remove an event on the calendar}
				\item{\textbf{Preconditions:} Officer is logged onto to the group's officer page}
				\item{\textbf{Summary:} Related use cases: Add Event, modify}
				\item{\textbf{Steps:}
				\begin{enumerate}
					\item{Officer clicks Events button}
					\item{Officer clicks Remove Event button}
					\item{Officer sends notfications and has the site updated as needed}
					\item{Officer selects save and exit}
				\end{enumerate}
				}
				\item{\textbf{Alternatives:} Officer cancels out of use case by clicking on a different navigational area or clicking Cancel}
				\item{\textbf{Postconditions:} Event has been removed from calendar and site}
			\end{itemize}
			\subsubsection{Send Group Email}
			\begin{itemize}
				\item{\textbf{Actors:} Officer}
				\item{\textbf{Goals:} the officer will be able to send email to the group}
				\item{\textbf{Preconditions:} Officer is logged onto to the group's email page}
				\item{\textbf{Summary:} Related use cases: Add Event, modify event}
				\item{\textbf{Steps:}
				\begin{enumerate}
					\item{Officer clicks the compose email button}
					\item{Officer composes an email}
					\item{Officer selects which subgroup the email goes to}
					\item{Officer clicks the send button}
				\end{enumerate}
				}
				\item{\textbf{Alternatives:} Officer cancels out of use case by clicking on a different navigational area or clicking Cancel}
				\item{\textbf{Postconditions:} email has been spellchecked, and sent.  Officer is returned to the group's email page}
			\end{itemize}
			\subsubsection{Modify Website}
			\begin{itemize}
				\item{\textbf{Actors:} Officer}
				\item{\textbf{Goals:}  to generate a web site of information about a group}
				\item{\textbf{Preconditions:} Officer is logged onto to the group's officer page, website is already posted and linked to GUS}
				\item{\textbf{Summary:} Related use cases: }
				\item{\textbf{Steps:}
				\begin{enumerate}
					\item{Click Web Site button}
					\item{Click Web site options}
					\item{Modify information on website}
					\item{Click Preview button}
					\item{Select save and exit}

				\end{enumerate}
				}
				\item{\textbf{Alternatives:} Officer cancels out of use case by clicking on a different navigational area or clicking Cancel}
				\item{\textbf{Postconditions:} the website is published on the website, and automatically linked to from its supergroup
}
			\end{itemize}
		\subsection{System feature 3: User Menu}
         \subsubsection{Edit Email}
			\begin{itemize}
				\item{\textbf{Actors:} User}
				\item{\textbf{Goals:} user will be able to send emails to other GUS users}
				\item{\textbf{Preconditions:} user is logged onto GUS}
				\item{\textbf{Summary:} Related use cases: }
				\item{\textbf{Steps:}
				\begin{enumerate}
					\item{User clicks Email button}
					\item{User clicks on the Drafts folder}
					\item{User clicks on the email to be edited}
					\item{User edits email}
					\item{User sends email, and is returned to the drafts page}
				\end{enumerate}
				}
				\item{\textbf{Alternatives:} User cancels out of use case by clicking on a different navigational area or clicking Cancel}
				\item{\textbf{Postconditions:} Email is sent and user is returned to drafts page}
			\end{itemize}
         \subsubsection{Edit Major}
			\begin{itemize}
				\item{\textbf{Actors:} User}
				\item{\textbf{Goals:} user will be able to change personal attributes}
				\item{\textbf{Preconditions:} user is logged onto GUS}
				\item{\textbf{Summary:} Related use cases: }
				\item{\textbf{Steps:}
				\begin{enumerate}
					\item{User clicks MyProfile button}
					\item{User makes changes to personal fields}
					\item{User selects save and exit}
				\end{enumerate}
				}
				\item{\textbf{Alternatives:} User cancels out of use case by clicking on a different navigational area or clicking Cancel}
				\item{\textbf{Postconditions:} User's profile and personal page is updated, and user is returned to their home page}
			\end{itemize}
         \subsubsection{Edit Personal Page}
			\begin{itemize}
				\item{\textbf{Actors:} User}
				\item{\textbf{Goals:} user will be able to change appearance and content of personal page}
				\item{\textbf{Preconditions:} user is logged onto GUS}
				\item{\textbf{Summary:} Related use cases: Change Major}
				\item{\textbf{Steps:}
				\begin{enumerate}
					\item{User clicks MyPage button}
					\item{User edits personal information}
					\item{User selects privacy settings; to include which user's and groups they want seeing their information}
					\item{User selects what color theme / appearance they want for their page}
					\item{User selects what items and information they want on their page}
					\item{User previews their page and clicks save}
				\end{enumerate}
				}
				\item{\textbf{Alternatives:} User can move through page changes non-sequenctially or cancel out}
				\item{\textbf{Postconditions:} User's personal page is updated}
			\end{itemize}
         \subsubsection{Join Group}
			\begin{itemize}
				\item{\textbf{Actors:} User}
				\item{\textbf{Goals:} user will be able to request to join group}
				\item{\textbf{Preconditions:} user is logged onto GUS}
				\item{\textbf{Summary:} Related use cases: }
				\item{\textbf{Steps:}
				\begin{enumerate}
					\item{User enters group name}
					\item{User clicks on the join button}
					\item{User identifies which subgroups they want to join}
					\item{User selects save and exit}
				\end{enumerate}
				}
				\item{\textbf{Alternatives:} User cancels out of use case by clicking on a different navigational area or clicking Cancel}
				\item{\textbf{Postconditions:} user's request is auto approved, or request sent to group, user is added to group (if approved), and user is returned to the group's page}
			\end{itemize}
         \subsubsection{Leave Group}
			\begin{itemize}
				\item{\textbf{Actors:} User}
				\item{\textbf{Goals:} user will be able to leave group}
				\item{\textbf{Preconditions:} user is logged onto GUS}
				\item{\textbf{Summary:} Related use cases: }
				\item{\textbf{Steps:}
				\begin{enumerate}
					\item{User clicks MyGroups button}
					\item{User selects the group to drop}
					\item{User clicks the drop button}
				\end{enumerate}
				}
				\item{\textbf{Alternatives:} User cancels out of use case by clicking on a different navigational area or clicking Cancel}
				\item{\textbf{Postconditions:} User is dropped from the group, the group is removed from the user page and the user is returned to the MyGroups page}
			\end{itemize}
         \subsubsection{Send Email}
			\begin{itemize}
				\item{\textbf{Actors:} User}
				\item{\textbf{Goals:} user will be able to send emails to other GUS users}
				\item{\textbf{Preconditions:} user is logged onto GUS}
				\item{\textbf{Summary:} Related use cases: }
				\item{\textbf{Steps:}
				\begin{enumerate}
					\item{User clicks Email button}
					\item{User clicks on the Drafts folder}
					\item{User clicks on the email to be edited}
					\item{User edits email}
					\item{User sends email, and is returned to the drafts page}
				\end{enumerate}
				}
				\item{\textbf{Alternatives:} User cancels out of use case by clicking on a different navigational area or clicking Cancel}
				\item{\textbf{Postconditions:} Email is sent and user is returned to drafts page}
			\end{itemize}
         \subsubsection{Send Text}
			\begin{itemize}
				\item{\textbf{Actors:} User}
				\item{\textbf{Goals:} user will be able to send texts to other GUS users}
				\item{\textbf{Preconditions:} user is logged onto GUS}
				\item{\textbf{Summary:} Related use cases: }
				\item{\textbf{Steps:}
				\begin{enumerate}
					\item{User clicks text button}
					\item{User types in the recipient}
					\item{User types the message}
					\item{User clicks send}
					\item{User makes any required changes, recipient, spelling (if required) go to step 4}
				\end{enumerate}
				}
				\item{\textbf{Alternatives:} User cancels out of use case by clicking on a different navigational area or clicking Cancel}
				\item{\textbf{Postconditions:} text is sent and user is returned to previous page}
			\end{itemize}
	      \subsubsection{Join Event}
			\begin{itemize}
				\item{\textbf{Actors:} User}
				\item{\textbf{Goals:} user will be able to sign up for event}
				\item{\textbf{Preconditions:} user is logged onto GUS, and at the group's site}
				\item{\textbf{Summary:} Related use cases: }
				\item{\textbf{Steps:}
				\begin{enumerate}
					\item{User hovers over the calendar, the system displays the user's calendar next to the group calendar}
					\item{User hovers over the event to join, the system displays event details}
					\item{User right-clicks on the event, and selects will-be-there, or may-be-there}
					\item{User may enter a message}
				\end{enumerate}
				}
				\item{\textbf{Alternatives:} User cancels out of use case by clicking on a different navigational area or clicking Cancel}
				\item{\textbf{Postconditions:} User is added to the event's guest list, the event is added to the user's calendar, and the user is returned to the group page}
			\end{itemize}
         \subsubsection{Leave Event}
			\begin{itemize}
				\item{\textbf{Actors:} User}
				\item{\textbf{Goals:} user will be able to drop an event}
				\item{\textbf{Preconditions:} user is logged onto GUS}
				\item{\textbf{Summary:} Related use cases: }
				\item{\textbf{Steps:}
				\begin{enumerate}
					\item{User clicks on MyCalendar}
					\item{User right-clicks on the event}
					\item{User selects drop}
				\end{enumerate}
				}
				\item{\textbf{Alternatives:} User cancels out of use case by clicking on a different navigational area or clicking Cancel}
				\item{\textbf{Postconditions:} User is removed from the event's guest list, the event is removed from the user's calendar, and the user is returned to the calendar page}
			\end{itemize}
			\subsubsection{Sign Up}
			\begin{itemize}
				\item{\textbf{Actors:} User}
				\item{\textbf{Goals:} user will be able to sign up for GUS}
				\item{\textbf{Preconditions:} user is at GUS page}
				\item{\textbf{Summary:} Related use cases: Modify personal information, modify personal page}
				\item{\textbf{Steps:}
				\begin{enumerate}
					\item{User clicks on Sign Up}
					\item{User completes steps for modifying their page}
					\item{User selects groups to join}
				\end{enumerate}
				}
				\item{\textbf{Alternatives:} User cancels out of use case by clicking on a different navigational area or clicking Cancel}
				\item{\textbf{Postconditions:} User is added to GUS, and relevant users/groups are notified}
			\end{itemize}

			\subsection{System feature 4: Forum Module)}
			\subsubsection{Add / Post Topic}
			\begin{itemize}
				\item{\textbf{Actors:} User}
				\item{\textbf{Goals:} User will add a post or start a topic}
				\item{\textbf{Preconditions:} User is logged in, and at the group's forum page}
				\item{\textbf{Summary:} Related use cases: }
				\item{\textbf{Steps:}
				\begin{enumerate}
					\item{User clicks on topic to post to, or clicks AddTopic button}
					\item{User completes a post (the default title is the first line of the post)}
					\item{User clicks post button The system spell-checks the post, and truncates it if it's too long, it also validates the post, and returns an error message (if needed)}
					\item{If needed user makes corrections and repeats step 3}
				\end{enumerate}
				}
				\item{\textbf{Alternatives:} user cancels out of use case}
				\item{\textbf{Postconditions:} the post is sent to the group for validation, and the user is returned to the topic page}
			\end{itemize}
			\subsubsection{Remove / Post Topic}
			\begin{itemize}
				\item{\textbf{Actors:} User}
				\item{\textbf{Goals:} User will remove a post or topic}
				\item{\textbf{Preconditions:} User is logged in, and at the group's forum page}
				\item{\textbf{Summary:} Related use cases:}
				\item{\textbf{Steps:}
				\begin{enumerate}
					\item{User clicks on topic or the post to remove}
					\item{User clicks remove button}
				\end{enumerate}
				}
				\item{\textbf{Alternatives:} User cancels out of use case}
				\item{\textbf{Postconditions:} If user has the correct permissions, the post or topic is removed, otherwise it is not}
			\end{itemize}

			\subsubsection{Edit / Post Topic}
			\begin{itemize}
				\item{\textbf{Actors:} User}
				\item{\textbf{Goals:} User will edit a post or a topic}
				\item{\textbf{Preconditions:} User is logged in, and at the group's forum page}
				\item{\textbf{Summary:} Related use cases: add post/topic}
				\item{\textbf{Steps:}
				\begin{enumerate}
					\item{User right-clicks on topic or post to be edited}
					\item{User selects edit}
					\item{User modifies the post (the default title is the first line of the post)}
					\item{User clicks post button, the system spell-checks the post, and truncates it if it's too long, it also validates the post, and returns an error message (if needed)}
					\item{If needed user makes corrections and repeats step 3}
				\end{enumerate}
				}
				\item{\textbf{Alternatives:} user cancels out of use case}
				\item{\textbf{Postconditions:} the post is sent to the group for validation, and the user is returned to the topic page}
			\end{itemize}


\chapter{Requirements Traceability}
		\section{...}
			\subsection{...}
			\subsection{...}
			\subsection{...}
			\subsection{...}
			\subsection{...}
			\subsection{...}
		\section{...}
			\subsection{...}
			\subsection{...}
			\subsection{...}
			\subsection{...}
			\subsection{...}
			\subsection{...}
		\section{...}
			\subsection{...}
			\subsection{...}
			\subsection{...}
			\subsection{...}
			\subsection{...}
			\subsection{...}



\part{...}
%----------------------------------------------- back matter of the document
\appendix                                      % following chapters are appendixes
\chapter{...}
\chapter{...}
\end{document}
