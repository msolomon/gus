% hmk2 team guspy
\documentclass[12pt]{article}
\usepackage{hyperref}
\usepackage{verbatim}
\setlength{\topmargin}{-.5in}
\setlength{\textheight}{9in}
\setlength{\oddsidemargin}{.125in}
\setlength{\textwidth}{6.25in}

\begin{document}
%\title{LaTeX Typesetting By Example}
%\author{Phil Farrell\\
%Stanford University School of Earth Sciences}
%\renewcommand{\today}{November 2, 1994}
%\maketitle
\section *  {HW\#2 - guspy} 
Completed Goals for this week:
	\begin {itemize}
\item split group into subteams
\item create development plan complete with milestones (Gantt chart)
\item establish consistent development environment for each team member
\item develop plan for testing
\end{itemize}
\href{ganttchart.html}{Gantt Chart}
\\
\\
Subteams 
	\begin{description}
\item Groups/Users/Roles/Permissions
		\begin{itemize}
\item Joran
\item Chandler
\item Nathan
\item Lee
\item John
		\end{itemize}
\item Widgets
		\begin{itemize}
\item Mike
\item Max
\item Jacob
\item Sasha
\item Stephen
		\end{itemize}
	\end{description}
%
Test Plans
	\begin{description}
\item Plan for testing consists of creating tests using Django's built-in test classes. After the classes are largely finished (as shown in the Gantt Chart), we will begin testing of those classes. Each test will be documented using restructed text in the docstring.
		\begin {itemize}
\item example test :
%
\verbatiminput{django_example.txt}
\item the tests can be run from the command line as such
			\begin{itemize}
\item run every test in the project 
\\
\verb+./manage.py test+
\item run every test in the specific application
\\
\verb+./manage.py test animals+
\item run an individual test case
\\
\verb+./manage.py test animals.AnimalTestCase+
			\end {itemize}
	\end {itemize}
	\end{description}
\end{document}