\documentclass[12pt]{article}
\usepackage{fullpage}
\begin{document}
\title{CS383 - Homework 5}
\author{Scott Beddall}
\maketitle

\section{Introduction}
\subsection{The Goal}
The goal of this assignment is to define a specific test case in isolation from GUS itself. To this end, I wrote a short java program to parse different usernames and passwords from a file and validate their content.
\subsection{The Method}
It is known login verification is a much needed security measure when dealing with any secure system. Because this is the case, I structured my test cases around whhat most standard login pages will accept. \\\\
It is much easier to define acceptable usernames and passwords than it is to define what ISN'T acceptable. To this end, I believe that the two inputs must follow these validation rules.

\begin{itemize}
\item Username Verification
  \begin{itemize}
    \item At least 6 characters long
    \item At most 20 characters long
    \item Accepts input of the set [a-z,A-Z,',-], rejects everything else
  \end{itemize}
\item Password Verification
  \begin{itemize}
    \item At least 5 characters long
    \item At most 20 characters long
    \item Accepts any input except for white space (tabs,spaces, etc)
  \end{itemize}
\end{itemize}

\section{Test Case}
The test case for login verification has two inputs: Username and Password. If both of these are validated successfully then a test is successful.\\\\
I will be using the test-driven-development model that Dr. J has suggested to the class.
\subsection{Name}
Login Verification
\subsection{Location of Files}
(From GUS main directory)\\
/doc/GROUP2/hw5/test.java\\
/doc/GROUP2/hw5/functions.java\\
/doc/GROUP2/hw5/test.dat\\
/doc/GROUP2/hw5/output.txt
\subsection{Input}
The input will be coming from a file named "test.dat". This file is made up usernames and passwords seperated by line of the form:\\
\emph{username}\\
\emph{password}\\
\emph{username}\\
\emph{password}\\
\emph{...}
\subsection{Oracle}
As the input file is parsed, each username and password is verified by functions present in "functions.java". As lines are scanned in, they are verified by the
associated functions. (isUserValid(),isPassValid())\\
If the functions find an issue with the input, they will output that specific problem, and output "Failed". If all tests are passed, "Successful" will be sent to standard output.
This process is shown in "output.txt".
\subsection{Log}
The predicted output is present in the file "output.txt" present in the folder in \emph{/doc/GROUP2/hw5/.}
\end{document}
